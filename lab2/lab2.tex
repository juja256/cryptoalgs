\documentclass[a4paper,12pt]{article}
\usepackage[T1]{fontenc}
\usepackage[dvipsnames]{xcolor}
\usepackage[utf8x]{inputenc}
\usepackage[russian]{babel}
\usepackage{pdflscape}
\usepackage{fancyvrb}
\usepackage{geometry}
\usepackage{indentfirst}
\usepackage{wrapfig}
\usepackage{placeins}
\usepackage{graphicx}
\usepackage{seqsplit}
 \geometry{
 a4paper,
 total={210mm,297mm},
 left=10mm,
 right=20mm,
 top=10mm,
 bottom=15mm,
 }

\usepackage{listings}
\lstset{basicstyle=\ttfamily,
  showstringspaces=false,
  commentstyle=\color{OliveGreen},
  keywordstyle=\color{MidnightBlue},
  extendedchars=\true,
  inputencoding=utf8x,
  breaklines=true,
  basicstyle=\ttfamily\footnotesize
}

\RecustomVerbatimCommand{\VerbatimInput}{VerbatimInput}%
{fontsize=\footnotesize,
 %
 %frame=lines,  % top and bottom rule only
 %framesep=2em, % separation between frame and text
 rulecolor=\color{Gray},
 %
 %label=\fbox{\color{Black}text2},
 %labelposition=topline,
}
\begin{document}

\begin{titlepage}
  \begin{center}
    \large
    Міністерство освіти і науки України
    
    Національний технічний університет України

    \textit{“Київський політехнічний інститут”}
    
    Фізико-технічний інститут
    \vspace{5cm}

    \textsc{Лабораторна робота №2}\\[5mm]
    
    {\LARGE Криптоаналіз афінної підстановки біграм}\\
  \bigskip
    
    
\end{center}
\vspace{3cm}
\hfill
\begin{minipage}{0.3\textwidth}
\large
  Виконав:\\
  \textit{Грубіян Є.О.}\\
  Прийняв:\\
  \textit{Яковлєв С.В.}\\ % aka Euler
  
\end{minipage}
\bigskip

\vfill
\vfill
\vfill
\begin{center}
  Київ 2015 р.
\end{center}

\end{titlepage}
\section*{Вхідні дані}

{\large 5 Варіант }
\VerbatimInput{05.txt}

\section*{Скрипт криптоаналізу}
\lstinputlisting[language=python]{affine.py}

\newpage
\section*{Дешифрований текст}
\VerbatimInput{out.txt}

\section*{Висновок}
В лабораторній роботі було зроблено криптоаналіз шифру біграмної підстановки та занйдено відкритий текст з поміж 2х кандидатів.
\end{document}
